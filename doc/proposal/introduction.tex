\section{Introduction}

Software systems today are large and incomprehensible.
This is due in part to the difficulty
of interfacing with computer hardware,
a large part of which is the translation
of high-level programming languages to the
low-level machine code of a CPU. This task is
made even more difficult by the complex and idiosyncratic
CPU architectures and operating systems of modern times.

We propose the use of a simple, machine-level
intermediate language to help make
the process of machine code generation
more understandable.
The language would be independent of any particular
source language or target machine.
Code in the language is meant to be
produced by a compiler, not written by a person.
Thus, the language is in the same
spirit as the original UNCOL (UNiversal
Computer Oriented Language)~\cite{strong1958problem}

Our primary design principle is conceptual parsimony.
This is because our overarching goal, comprehension,
is frustrated by conceptual complexity.

\subsection{Advantages}

\textbf{Breaking down the machine code generation task} makes
it more manageable. Compilers can focus on lowering
the semantics of a program to the machine level,
without concern for the details of a particular CPU or operating system.
At the same time, actual machine code generation can be done
by a separate software component that can focus only on
register allocation, instruction selection, instruction
encoding, and the platform ABI.

\textbf{Increased portability} is an obvious benefit to using an intermediate language.
Programming languages can be compiled to multiple machine architectures
via a single intermediate target language.
It is our intention to use our intermediate language
to run programs on the x86 and ARM architectures and
the Mac OS X, Windows and Linux operating systems (in their more recent forms).

\textbf{Effective visualization of low-level system behavior} can be a tremendous help
not only to those first learning the lower-layers of a system, but
also to those actively working on them.
A simplified intermediate model of machine-level computation is
essential to making this happen.

\textbf{Emulation opportunities} include virtualizing computation,
easier software bootstrapping, and greater ease of tracking down errors
in the compilation chain.

\subsection{Not Goals}

To understand the language design, it is also helpful to know the
things that are not goals.

\textbf{Efficiency is not a goal}.
Most intermediate languages for machine code generation
are designed with efficiency in mind.
This encompasses efficiency of execution, efficiency
of memory use, efficiency of compilation, etc.
We deliberately eschew these concerns
in our design because the simplest designs
are much less complex than the more efficient designs, generally.
This does not rule out, of course, that optimizations be applied before
code is generated in our language. Nor does it rule out that a translator of
our language to machine code attempt to produce optimized
code, similar to what what has been done~\cite{davidson1980design, tanenbaum1982using}.
The point is that we endeavor to keep efficiency concerns from influencing our
design so that our primary objectives are met.

\textbf{Wide portability is not a goal.}
Although portability is an advantage of the language, we intent
to support only the few machine architectures and operating systems
mentioned above. Limiting the target platforms allows us to make
many simplifying assumptions in our design.

\textbf{Helping people write assembly code directly is not a goal.}
Some assembly languages provide features for this.
Adding such features would only complicate our language
and goes against the intention that code in the language be
produced by a compiler. Thus, there are no macro facilities, high-level commands,
or pseudo-instructions to make it easier for a person to write code.

\textbf{Generality or extensibility is not a goal.}
For similar reasons, we do not strive to produce a highly general
or extensible design. As long as the virtual machine provides
a convenient path to the fundamental machine instructions, other forms of
computation can be implemented by programming language compilers in
terms of these instructions.

\documentclass[10pt]{article}
\usepackage[top=1in, left=1in, right=1in, bottom=1in]{geometry}
\usepackage[numbers]{natbib}
\begin{document}
\twocolumn

\title{The UNCOL I Never Had}
\author{Dan Amelang (dan@amelang.net)}
%\date{}
\maketitle

\begin{quotation}\small
``One of the fundamental problems facing the computer profession today is the considerable
length of time required to develop an effective method of communication with the machine."

-- Strong et al. (1958)
\end{quotation}

\begin{abstract}

The following is a proposal for a computer-oriented intermediate language.
The language is designed to assist in creating understandable
software systems by providing a simple, low-level target for
compilation, emulation and visualization.
Underlying the language is a
minimal RISC-based virtual machine with an unlimited number of typed registers.
Both the instruction set and the assembly-like language syntax are fairly
traditional, though an effort was made to streamline both.
A small amount of additional syntax and pseudo-instructions were added to achieve
portability across machine architectures and operating systems.

\end{abstract}

\section{Introduction}

Software systems today are large and incomprehensible.
This is due in part to the difficulty
of interfacing with computer hardware,
a large part of which is the translation
of high-level programming languages to the
low-level machine code of a CPU. This task is
made even more difficult by the complex and idiosyncratic
CPU architectures and operating systems of modern times.

We propose the use of a simple, machine-level
intermediate language to help make
the process of machine code generation
more understandable.
The language would be independent of any particular
source language or target machine.
Code in the language is meant to be
produced by a compiler, not written by a person.
Thus, the language is in the same
spirit as the original UNCOL (UNiversal
Computer Oriented Language)~\cite{strong1958problem}

Our primary design principle is conceptual parsimony.
This is because our overarching goal, comprehension,
is frustrated by conceptual complexity.

\subsection{Advantages}

\textbf{Breaking down the machine code generation task} makes
it more manageable. Compilers can focus on lowering
the semantics of a program to the machine level,
without concern for the details of a particular CPU or operating system.
At the same time, actual machine code generation can be done
by a separate software component that can focus only on
register allocation, instruction selection, instruction
encoding, and the platform ABI.

\textbf{Increased portability} is an obvious benefit to using an intermediate language.
Programming languages can be compiled to multiple machine architectures
via a single intermediate target language.
It is our intention to use our intermediate language
to run programs on the x86 and ARM architectures and
the Mac OS X, Windows and Linux operating systems (in their more recent forms).

\textbf{Effective visualization of low-level system behavior} can be a tremendous help
not only to those first learning the lower-layers of a system, but
also to those actively working on them.
A simplified intermediate model of machine-level computation is
essential to making this happen.

\textbf{Emulation opportunities} include virtualizing computation,
easier software bootstrapping, and greater ease of tracking down errors
in the compilation chain.

\subsection{Not Goals}

To understand the language design, it is also helpful to know the
things that are not goals.

\textbf{Efficiency is not a goal}.
Most intermediate languages for machine code generation
are designed with efficiency in mind.
This encompasses efficiency of execution, efficiency
of memory use, efficiency of compilation, etc.
We deliberately eschew these concerns
in our design because the simplest designs
are much less complex than the more efficient designs, generally.
This does not rule out, of course, that optimizations be applied before
code is generated in our language. Nor does it rule out that a translator of
our language to machine code attempt to produce optimized
code, similar to what what has been done~\cite{davidson1980design, tanenbaum1982using}.
The point is that we endeavor to keep efficiency concerns from influencing our
design so that our primary objectives are met.

\textbf{Wide portability is not a goal.}
Although portability is an advantage of the language, we intent
to support only the few machine architectures and operating systems
mentioned above. Limiting the target platforms allows us to make
many simplifying assumptions in our design.

\textbf{Helping people write assembly code directly is not a goal.}
Some assembly languages provide features for this.
Adding such features would only complicate our language
and goes against the intention that code in the language be
produced by a compiler. Thus, there are no macro facilities, high-level commands,
or pseudo-instructions to make it easier for a person to write code.

\textbf{Generality or extensibility is not a goal.}
For similar reasons, we do not strive to produce a highly general
or extensible design. As long as the virtual machine provides
a convenient path to the fundamental machine instructions, other forms of
computation can be implemented by programming language compilers in
terms of these instructions.

\section{Virtual Machine}

\textbf{Underlying our proposed language is a
minimal virtual machine that follows the RISC philosophy}~\cite{patterson1985reduced}.
This means that individual machine instructions
perform only basic units of work.
It also implies a register-based architecture that
is ``load-store", meaning that memory access is limited to
\texttt{LOAD} and \texttt{STORE} instructions.
Finally, it implies that the memory addressing modes of the \texttt{LOAD} and
\texttt{STORE} instructions are few and simple.

This design was chosen because the operation of
RISC machines is conceptually straightforward. In addition,
generating ARM and x86 code from RISC code is not difficult.
ARM is basically a RISC architecture, and though x86 is not a RISC architecture,
a subset of its instruction set can be targeted as if it were.

Like physical machines, \textbf{the virtual machine processes
linear sequences of instructions} and uses jump or call instructions
to alter the flow of control. This is in contrast with ``structured programming,"
with code organized in blocks or as a tree structure.
As is the case with x86 and ARM CPUs, \textbf{conditional branching depends on flags in
a status register}. For portability reasons, the status register
of the virtual machine cannot be read or modified directly.

Also like many physical machines, \textbf{our virtual machine requires that data be
properly aligned}. No automatic alignment is provided.
In addition, \textbf{the endianness of the virtual machine is not defined.}
Layout of multi-byte primitive data types depends on the underlying machine.

Because the virtual machine is very similar to modern hardware,
\textbf{machine code generated from our language should not require a
runtime library or initialization routines}. An exception to this is necessary
when a fundamental operation (e.g., integer division) is not available as
an instruction (or as a short sequence of instructions) on a particular
architecture (e.g., ARM).  In this case, a function call to a runtime
library is needed. This library could be from a third-party (e.g., libgcc).

\subsection{Registers}

Physical machines have a limited number of registers, and this number
can vary significantly from architecture to architecture.
Rather than commit to a certain number of registers, \textbf{our virtual
machine provides an unlimited number of registers}. These registers
may or may not map directly to actual machine registers.

Although single static assignment form is convenient for performing
optimizing code transformations, we chose instead to follow physical machine
register semantics. Thus, \textbf{our virtual machine allows
registers to be assigned a value multiple times.}

Modern CPUs segregate register banks by data type.
For example, modern machines have registers for memory address values
that are separate from the registers for floating point values.
But there are also several cases where data types will share the same bank.
For example, registers for memory address values are also used for signed integer
values. This is also the case for floating point values and integer SIMD vectors.

To avoid this complication,
\textbf{our virtual machine provides a separate register bank for
each primitive data type}.
It is left to the machine code generator to select the
appropriate register bank of the actual machine in each case.

\subsection{Machine Word Width}
\label{sec:word}

Physical machines have a natural ``word" data type that is wide enough
for working with memory address values. For this same purpose,
our virtual machine has word registers. But, because the width of this data type
varies from machine to machine, \textbf{the width of word registers in
our virtual machine is undefined.}

Unfortunately, it is now unclear how much
memory should be allocated for storage of word values.
We resolve this problem by specifiying that
\textbf{word values stored in memory are assumed to occupy 64-bits.}
This leaves enough room for word values of all modern machines.
Optimizing storage of word values in memory on 32-bit machines
is not a concern.

\subsection{Instructions}

Some instruction sets do not always separate source operands from
destination operands (i.e., a source operand must double as the destination
in some cases).
This can make the machine a more difficult compilation target.
Thus, \textbf{source and destination operands are separate} in our
virtual machine instructions.
Of course, it is possible to use a register as both a source
and destination operand in the same instruction.

Rather than define or adopt a specific ABI, virtual machine \textbf{pseudo-instructions
provide platform-specific functionality}.
For example, function calling is provided by a pseudo-instruction because
it requires knowing how arguments are passed on the call stack.
An important advantage of implementing the platform ABI is that it allows for interfacing
with machine code that was not compiled via our virtual machine abstraction.

A description of the instruction set of the virtual machine
is given in Section 4.

\section{Syntax}

This section contains an informal description
of a syntax for our proposed language.
The basic appearance of the language is close
to that of traditional assembly languages.
Program text is restricted to the ASCII
character encoding. All constructs are case-\emph{\textbf{in}}sensitive,
with the exception of labels.

\subsection{Literals}

Integer literals are written in hexadecimal format, with a leading
\textbf{\texttt{0x}}. The width (i.e., size) of the literal value is
inferred from the number of characters. Literals can be 8-bits, 16-bits,
32-bits or 64-bits wide. This syntax is used for both signed and unsigned values.
In the generated code, the byte ordering of multi-byte values follows the endianness
of the target platform.

\begin{table}[h!]
  \begin{center}
    \begin{tabular}{|r|l|}
      \hline
      \texttt{0x01}               & \ 8-bit integer  \\ \hline
      \texttt{0x0123}             &   16-bit integer \\ \hline
      \texttt{0x01234567}         &   32-bit integer \\ \hline
      \texttt{0x0123456789ABCDEF} &   64-bit integer \\ \hline
    \end{tabular}
  \end{center}
  \caption{Example integer literals}
\end{table}

Floating point literals are written in decimal format. Single-precision literals
are written with a single decimal point, while double-precision
literals with double decimal points.

\begin{table}[h!]
  \begin{center}
    \begin{tabular}{|r|l|}
      \hline
      \texttt{12.34}  & Single-precision floating point \\ \hline
      \texttt{12..34} & Double-precision floating point \\ \hline
    \end{tabular}
  \end{center}
  \caption{Example floating point literals}
\end{table}

\subsection{Labels}

Labels are aliases to addresses in memory.
Labels are identified by
one or more letters, numbers or symbols. The two
permitted symbols are {\LARGE \textbf{\texttt{\_}}} and \textbf{\texttt{\$}}.
Labels are case-sensitive and cannot begin with a digit.
Local labels must be prefixed with a \textbf{\texttt{.}},
and can only be referenced within the function in which they are defined.
Global labels must be prefixed with a \textbf{\textbf{@}}
and can be referenced from any part of the program.

\subsection{Global data}

Global data is declared using a global label and a colon, followed by one or more
literals. Literals are separated by whitespace. The amount of memory to allocate
is equal to the sum width of the literals.

\begin{verbatim}
@hello_world_string:
    0x68 0x65 0x6c 0x6c 0x6f 0x20
    0x77 0x6f 0x72 0x6c 0x64 0x00
\end{verbatim}

In the above example, ``hello\_world\_string" is a global label that refers
to the beginning of a 12-byte section of memory that contains
the ASCII characters of the string ``hello world", ending with the \texttt{NULL} terminator.

\subsection{Registers}

Register references are composed of one or more characters followed by an integer
ranging from $1$ to $n$. The characters correspond to the type of the register, as shown below.

\begin{table}[h!]
  \begin{center}
    \begin{tabular}{|l|l|}
      \hline
      b & 8-bit integer         \\ \hline
      s & 16-bit integer        \\ \hline
      i & 32-bit integer        \\ \hline
      l & 64-bit integer        \\ \hline
      w & word-size integer     \\ \hline
      f & 32-bit floating point \\ \hline
      d & 64-bit floating point \\ \hline
    \end{tabular}
  \end{center}
  \caption{Register type identifiers}
\end{table}

The word type represents the natural data type of the underlying hardware.
Registers of this type can hold memory addresses, and address arithmetic
can be performed using instructions that operate on these registers.
See Section~\ref{sec:word} for more details about the word type.

A separate register bank exists for each register type. The integer
part of the register reference refers to a specific register
within the bank of that type. Register references start with 1.

\begin{table}[h!]
  \begin{center}
    \begin{tabular}{|l|l|}
      \hline
      \texttt{b1}  & First 8-bit register                  \\ \hline
      \texttt{d11} & Eleventh double-precision FP register \\ \hline
      \texttt{w4}  & Fourth word register                  \\ \hline
    \end{tabular}
  \end{center}
  \caption{Example register references}
\end{table}

\subsection{Instruction Statements}

Instruction statements begin with an instruction name, zero or
more source operands, and optionally a right arrow followed by
zero of more destination operands. Source operands can be either registers,
literals, or labels. Destination operands must be registers.
A register can occur more than once as an operand in a statement.

To avoid a combinatorial explosion of instruction names,
we allow instructions to be polymorphic over operand type.
``Type" here refers to register, literal or label as well
as data type such as 32-bit floating point.
This way, for example, \texttt{ADD f1 f2 -> f2} can be shorthand
for \texttt{ADD.F f1 f2 -> f2}.

Not all instructions can be applied to all types.
A more complete specification of our language would include
a full listing of the permitted data types of each instruction
(e.g., \texttt{ADD $\alpha$ $\alpha$ -> $\alpha$}). Section 4 gives an informal description of each instruction
and its permitted operand types.

Example instruction statements:

\begin{verbatim}
    ADD   b1 0x01 -> b1
    STORE i4 w5
    JUMP  .label2
\end{verbatim}

\subsection{Local labels}

A local label refers to the address of an instruction.
The label is written before the instruction statement.

\begin{verbatim}
.loop:   XOR s1 s2 -> s1
         DIV f3 f2 -> f1
.L21:
         LOAD w5 -> l1
\end{verbatim}

\subsection{Comments}

Comments begin with a semicolon and continue until the end of the line.

\begin{verbatim}
    ADD w1 0x4 -> w1 ; Increment the pointer
\end{verbatim}

\subsection{Function Definition}

Function definition, including argument passing, must be abstracted to avoid
the complications and variations of platform calling conventions.

Functions are defined by a global label, a parenthesized list of zero
or more arguments, and a colon. Arguments are received in registers.
The body of the function is given by the instruction statements that follow
until a global label is encountered or end-of-file.

\begin{verbatim}
@multiply (i1 i2):
    MUL i1 i2 -> i1
    RETURN i1

@getName ():
    LOAD @name -> w1
    RETURN w1
\end{verbatim}

Custom calling conventions and stack management may
be possible with future additions to the language.

\section{Instruction Set}

This section presents a mostly complete description
of the instruction set of the virtual machine.
We leave the binary encoding of instructions unspecified.

\subsection{Arithmetic and Bitwise Operations}

Arithmetic and bitwise instructions have two source operands and
one destination operand. All operands must be the same data type.
The source operands must be two registers or one register and one literal.
Bitwise and unsigned arithmetic instructions cannot be applied to
floating point registers. For a list of these instructions,
see Table \ref{ArithmeticAndBitwiseInstructions}.

\begin{table}[h!]
  \begin{center}
    \begin{tabular}{|l|l|}
      \hline
      ADD  & Addition               \\ \hline
      SUB  & Subtraction            \\ \hline
      MUL  & Multiplication         \\ \hline
      DIV  & Division               \\ \hline
      UDIV & Unsigned division      \\ \hline
      REM  & Remainder              \\ \hline
      UREM & Unsigned remainder     \\ \hline
      AND  & Logical conjunction     \\ \hline
      OR   & Logical disjunction    \\ \hline
      XOR  & Exclusive disjunction  \\ \hline
      SHL  & Bit shift left         \\ \hline
      SHR  & Bit shift right        \\ \hline
      ASHR & Arithmetic shift right \\ \hline
    \end{tabular}
  \end{center}
  \caption{Arithmetic and Bitwise Instructions}
  \label{ArithmeticAndBitwiseInstructions}
\end{table}

Example instructions:

\begin{verbatim}
    ADD i1 i2   -> i1
    MUL f5 0.5  -> f6
    AND b3 0x0F -> b3
\end{verbatim}

\subsection{Copying to a Register}

The \texttt{COPY} instruction is used to copy a value into
a register. The source operand may be another register,
a literal value, or a label, in which case the address
the label refers to is copied into the destination operand.
The destination operand must have the same type as the source operand.
The \texttt{COPY} instruction does not read or write from memory.

\begin{verbatim}
    COPY i1    -> i2
    COPY 0x00  -> b9
    COPY @var1 -> w2 ; Copy address of @var1
\end{verbatim}

\subsection{Memory Access}

The \texttt{LOAD} instruction reads the value in memory
at the address given by the source operand. The source operand
must be a word register or a label. The data type of the
destination register determines how much is read from memory.

\begin{verbatim}
    LOAD w3    -> s1 ; Load 16-bit value
    LOAD @var1 -> w1 ; Load word at @var1
\end{verbatim}

The \texttt{STORE} instruction writes the value given by
the first operand into the memory address given by the
second operand. The second operand must be a word register
or a label.

\begin{verbatim}
    STORE i2 w6   ; Write 32-bits at w6
    STORE 0x00 w1 ; Store a byte at w1
\end{verbatim}

\subsection{Control Flow}

An unconditional branch is performed using the \texttt{JUMP} instruction, which
takes a local label its only operand.

\begin{verbatim}
    JUMP .loop
\end{verbatim}

The \texttt{COMP} instruction compares two values and sets condition flags
accordingly in an implicit status register. Subsequent conditional branch instructions of the
form \texttt{JUMP*} use the flags to determine program flow.

\begin{verbatim}
    COMP  f4 1.0
    JUMP< .label3 ; Jump if f4 < 1.0
\end{verbatim}

\subsection{Conversion}

Conversion between data types is provided by the
\texttt{CONV} and \texttt{UCONV} instruction.

\begin{verbatim}
    CONV  i3 -> f1 ; Signed int to float
    UCONV i3 -> f2 ; Unsigned int to float
\end{verbatim}

\subsection{Stack Memory Allocation}

The \texttt{ALLOC} pseudo-instruction allocates memory
on the call stack. The source operand
specifies the amount of memory to allocate in bytes,
and must be a literal integer. The destination operand, which must
be a word register, receives the start address of the
allocated memory.

\begin{verbatim}
    ALLOC 0x1 -> w4 ; Allocate 1 byte on stack
\end{verbatim}

Successive calls to \texttt{ALLOC} result in adjacent
allocated portions of memory (though stack growth direction
is unspecified). At function start, the stack pointer
is guaranteed to be 128-bit aligned. But, the assembly
code must pad allocated stack memory as needed for alignment.
This is because the implicit stack pointer of
the virtual machine only moves by the amount given to \texttt{ALLOC}.

\subsection{Function Calling}

The \texttt{CALL} pseudo-instruction provides a platform-independent
mechanism for calling a function. The first operand must be
a global label or a word register. Zero or more function arguments follow,
all of which must be registers. Return values, if any, are received
by destination registers.
The assembler does not check that argument types match those
in the function definition.

\begin{verbatim}
    CALL @getNameForId i3 -> w2
    CALL @puts w2
\end{verbatim}

The \texttt{RETURN} pseudo-instruction is used to return program
control, and optionally values also, to the calling function.

\begin{verbatim}
    RETURN d2
\end{verbatim}

In the future, an additional pseudo-instruction \texttt{SYSCALL}
may be defined for interfacing with the operating system.
For functional programming, a \texttt{TAILCALL} pseudo-instruction
could also be defined.

\subsection{Platform-Specific Data Layout}

In some cases, interfacing with code generated by a C compiler
requires laying out data in memory according to the
underlying machine word width and operating system ABI.
This need arises when calling a C function that accepts
a pointer to a structure as an argument.
To this end, the \texttt{PACK} pseudo-instruction takes a list
of source registers, allocates an appropriate amount of stack memory,
and copies the register values to the allocated space according
to rules of the target platform. The \texttt{UNPACK} pseudo-instruction
does the reverse.

\begin{verbatim}
    PACK b4 w2 i1 -> w5
    CALL @func w5
    UNPACK w5 -> b4 w2 i1
\end{verbatim}

\section{Hello World Example}

The following is an example of a small program written in our proposed language.
First, the ASCII string ``hello world" is declared in global memory
with the label ``hello\_string".
Then, the function \texttt{main} is declared as the entry point
into the program. The instructions in \texttt{main} first load the address
of the string, then call the \texttt{puts} function with the address
(which should print the string). Finally, the number zero is loaded
into a register and returned.
\begin{verbatim}
@hello_string:
  0x68 0x65 0x6c 0x6c 0x6f 0x20
  0x77 0x6f 0x72 0x6c 0x64 0x00

@main ():
  LOAD @hello_string -> w1
  CALL @puts w1
  LOAD 0x00 -> i1
  RETURN i1
\end{verbatim}
For comparison, we include an equivalent program in the LLVM assembly language~\cite{lattner2008llvm}:
\begin{verbatim}
@hello_string = internal constant [12 x i8] c"hello, world\00"

declare i32 @puts(i8*)

define i32 @main() {
  %0 = call i32 @puts(i8* getelementptr inbounds
                      ([12 x i8]* @hello_string, i32 0, i32 0))
  ret i32 0
}
\end{verbatim}

%Organize/group this

* UNCOL
* ZCODE
* OCODE (INTCODE, CINTCODE)
* PL360
* Janus
* EM (from Amsterdam Compiler Kit)
* TDF (TenDRA Distribution Format -- part of the ANDF effort)
* MLRISC
* C--
* LLVM
* Java bytecode
* P-code (pascal)
* CIL
* C minor/RTL/LTL/Linear (in the Verified Software Toolchain)
* teaching languages
* TALs?

Many languages were designed with a specific source language or machine target,
This de-emphasis of efficiency sets our language apart from almost all others in this class.

None (?) provide a (open source), machine-level (machine model),
simplicity (design not complicated by optimization concerns),
source language neutral,
target-machine neutral.

It should be noted that once effeciency is considered
a secondary goal, most previous work on intermediate languages
becomes only vagely/loosely/tenously relavent. retains
interest, relevance,.

Our design goals of simplicity, machine-level semantics, source language
independence and target machine independence, particularly at the sacrifice
of efficiency make asdf unique.

Our interests
are in an even lower level view of the hardware. Particularly, we are driven by a desire for
parsimony that such large systems do not support.

Generally, our effort distiguishes itself by targeting a small and similar group of target
architectures, very close/similar to real assembly language (not customizable, not trees), escewing optimization concerns,
and focusing primarily on building a component
this is itself small and simple, and can be used as a component for building a parsimonious
whole system that is parsimonious.

(be clear about how we differ from the literature)

As disscussed in SEction 1, our goals are ... To our knowledge, no previous
work on intermediate languages for code generation have had the same goals
(were driven by same goals, driven by rithless simpligity, nearly tot he exent that
we do)

(talk about C? dynamic code generation, complex, etc.)

(x86 has lots of old, legacy stuff we don't need)
(use SSE for floats, ignore the x87 floats)
(segmentation, addressing modes)

-----------

name, reference,
design goals
interesting implementation details

----------------

\section{Related Work}

Intermediate languages for machine code generation have been a topic of research
for over 50 years. We narrow the scope of this review by
focusing primarily on languages that were designed independent of source language
and independent of target machine. Also, we ignore languages designed
only for internal use within a specific compiler (e.g., GCC's GIMPLE language).

(we care about machine models, now!)


\bibliographystyle{plainnat}
\bibliography{bib}

\end{document}

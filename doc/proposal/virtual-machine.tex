\section{Virtual Machine}

Underlying our proposed language is a
minimal virtual machine that follows the RISC philosophy~\cite{patterson1985reduced}.
This means that individual instructions only
perform basic units of work rather than elaborate functions.
It also implies a register-based ``load-store" architecture,
which limits memory access to \texttt{LOAD} and \texttt{STORE} instructions.
Finally, it implies that the memory addressing modes of the \texttt{LOAD} and
\texttt{STORE} instructions are few and simple. This architecture was chosen
for our language because the operation of
of RISC machines is conceptually simple, making them
appealing targets for compilation, emulation, and visualization.

Like physical machines, the virtual machine does not support ``structured programming."
That is, program control is linearized -- not determined by blocks
or tree structures. And like x86 and ARM machines, conditional branching
depends on flags in a status register, which for portability reasons
cannot be read or modified directly in our virtual machine.

Physical machines have a limited number of registers, and this number
can vary significantly from architecture to architecture.
Rather than commit to a certain number of registers, our virtual
machine provides an unlimited number of registers. These registers
may or may not map directly to actual machine registers.

Physical machines have separate register banks for different data types.
For example, modern machines have registers for memory addresses
that are separate from the registers for floating point values.
But there are also several cases where data types will share the same bank.
For example, registers for memory addresses are also used for signed integer
values. This is also the case for floating point values and integer SIMD vectors.

To avoid the complication of partially shared register banks,
our virtual machine provides a separate bank for each data type.
It is left to the final code generator to select the
appropriate register bank of the actual machine in each case.
Given the strict separation of the virtual registers,
it is only appropriate that we also make them strongly-typed.

Although single static assignment form is convenient for performing
optimizing code transformations, we chose instead to stay close to actual machine
register semantics, which allows the same register to be assigned a value multiple times.

Some instruction sets do not always separate source operands from
destination operands (i.e., a source operand may also double as the destination).
We chose to keep source and destination separate because this arrangement
is strictly more flexible. If desired, a source register can also be the
destination by merely referring to it twice in the same instruction statement.
We believe that this design makes our language an easier compilation target.

Physical machines have a natural ``word" data type, used primarily
for working with memory addresses. Our virtual machine has a word register
type for this purpose. But, because the width of this data type
varies from machine to machine, we leave its register width undefined.
When stored in memory, though, we define the word width to be 64-bits.
This leaves room for the largest word width found in modern machines.
Optimizing storage of addresses in memory on 32-bit machines
is not a priority.

Automatic alignment of data in memory is not provided by the language,
though it is required for correct machine behavior.
In addition, the endianness of the virtual machine is not defined. This behavior
is determined by the underlying machine or is chosen by the emulator.

Rather than define or adopt a specific ABI, the virtual machine provides a
handful of pseudo-instructions that map to platform-specific functionality.
For example, allocating call stack memory is provided by a pseudo-instruction
because it requires knowing the direction that the stack grows.
Also, function calling is provided by a pseudo-instruction because
it requires knowing how arguments are passed on the call stack.
This approach allows for interfacing with machine code not compiled
via our virtual machine abstraction.

Because the virtual machine is very similar to modern hardware,
machine code generated from our language should not require a
runtime library. An exception to this is necessary when a fundamental operation
(e.g., integer division) is not available as an instruction (or as a short
sequence of instructions) on a particular CPU architecture (e.g., ARM).
In this case, a function call to a runtime library is needed.
This library could be from a third-party (e.g., libgcc) or it could
be our own.

A description of the instruction set of the virtual machine
is given in Section 4.

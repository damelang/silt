\section{Syntax}

This section contains an informal description
of a syntax for our proposed language.
The basic appearance of the language is close
to that of traditional assembly languages.
Program text is restricted to the ASCII
character encoding. All constructs are case-\emph{\textbf{in}}sensitive,
with the exception of labels.

\subsection{Literals}

Integer literals are written in hexadecimal format, with a leading
\textbf{\texttt{0x}}. The width (i.e., size) of the literal value is
inferred from the number of characters. Literals can be 8-bits, 16-bits,
32-bits or 64-bits wide. This syntax is used for both signed and unsigned values.
In the generated code, the byte ordering of multi-byte values follows the endianness
of the target platform.

\begin{table}[h!]
  \begin{center}
    \begin{tabular}{|r|l|}
      \hline
      \texttt{0x01}               & \ 8-bit integer  \\ \hline
      \texttt{0x0123}             &   16-bit integer \\ \hline
      \texttt{0x01234567}         &   32-bit integer \\ \hline
      \texttt{0x0123456789ABCDEF} &   64-bit integer \\ \hline
    \end{tabular}
  \end{center}
  \caption{Example integer literals}
\end{table}

Floating point literals are written in decimal format. Single-precision literals
are written with a single decimal point, while double-precision
literals with double decimal points.

\begin{table}[h!]
  \begin{center}
    \begin{tabular}{|r|l|}
      \hline
      \texttt{12.34}  & Single-precision floating point \\ \hline
      \texttt{12..34} & Double-precision floating point \\ \hline
    \end{tabular}
  \end{center}
  \caption{Example floating point literals}
\end{table}

\subsection{Labels}

Labels are aliases to addresses in memory.
Labels are identified by
one or more letters, numbers or symbols. The two
permitted symbols are {\LARGE \textbf{\texttt{\_}}} and \textbf{\texttt{\$}}.
Labels are case-sensitive and cannot begin with a digit.
Local labels must be prefixed with a \textbf{\texttt{.}},
and can only be referenced within the function in which they are defined.
Global labels must be prefixed with a \textbf{\textbf{@}}
and can be referenced from any part of the program.

\subsection{Global data}

Global data is declared using a global label and a colon, followed by one or more
literals. Literals are separated by whitespace. The amount of memory to allocate
is equal to the sum width of the literals.

\begin{verbatim}
@hello_world_string:
    0x68 0x65 0x6c 0x6c 0x6f 0x20
    0x77 0x6f 0x72 0x6c 0x64 0x00
\end{verbatim}

In the above example, ``hello\_world\_string" is a global label that refers
to the beginning of a 12-byte section of memory that contains
the ASCII characters of the string ``hello world", ending with the \texttt{NULL} terminator.

\subsection{Registers}

Register references are composed of one or more characters followed by an integer
ranging from $1$ to $n$. The characters correspond to the type of the register, as shown below.

\begin{table}[h!]
  \begin{center}
    \begin{tabular}{|l|l|}
      \hline
      b & 8-bit integer         \\ \hline
      s & 16-bit integer        \\ \hline
      i & 32-bit integer        \\ \hline
      l & 64-bit integer        \\ \hline
      w & word-size integer     \\ \hline
      f & 32-bit floating point \\ \hline
      d & 64-bit floating point \\ \hline
    \end{tabular}
  \end{center}
  \caption{Register type identifiers}
\end{table}

The word type represents the natural data type of the underlying hardware.
Registers of this type can hold memory addresses, and address arithmetic
can be performed using instructions that operate on these registers.
Although the actual width of a word is machine-dependent,
we define the width of a word, when stored in memory, to be 64-bits.

A separate register bank exists for each register type. The integer
part of the register reference refers to a specific register
within the bank of that type. Register references start with 1.

\begin{table}[h!]
  \begin{center}
    \begin{tabular}{|l|l|}
      \hline
      \texttt{b1}  & First 8-bit register               \\ \hline
      \texttt{d11} & Eleventh double-precision register \\ \hline
      \texttt{w4}  & Fourth word register               \\ \hline
    \end{tabular}
  \end{center}
  \caption{Example register references}
\end{table}

\subsection{Instruction Statements}

Instruction statements begin with an instruction name, zero or
more source operands, and optionally a right arrow followed by
zero of more destination operands. Source operands can be either registers,
literals, or labels. Destination operands must be registers.
A register can occur more than once as an operand in a statement.

To avoid a combinatorial explosion of instruction names,
we allow instructions to be polymorphic over operand type.
``Type" here refers to register, literal or label as well
as data type such as 32-bit floating point.
This way, for example, \texttt{ADD f1 f2 -> f2} can be shorthand
for \texttt{ADD.F f1 f2 -> f2}.

Not all instructions can be applied to all types.
A more complete specification of our language would include
a full listing of the permitted data types of each instruction
(e.g., \texttt{ADD $\alpha$ $\alpha$ -> $\alpha$}). Section 4 gives an informal description of each instruction
and its permitted operand types.

Example instruction statements:

\begin{verbatim}
    ADD   b1 0x01 -> b1
    STORE i4 w5
    JUMP  .label2
\end{verbatim}

\subsection{Local labels}

A local label refers to the address of an instruction.
The label is written before the instruction statement.

\begin{verbatim}
.loop:   XOR s1 s2 -> s1
         DIV f3 f2 -> f1
.L21:
         LOAD w5 -> l1
\end{verbatim}

\subsection{Comments}

Comments begin with a semicolon and continue until the end of the line.

\begin{verbatim}
    ADD w1 0x4 -> w1 ; Increment the pointer
\end{verbatim}

\subsection{Function Definition}

Function definition, including argument passing, must be abstracted to avoid
the complications and variations of platform calling conventions.

Functions are defined by a global label, a parenthesized list of zero
or more arguments, and a colon. Arguments are received in registers.
The body of the function is given by the instruction statements that follow
until a global label is encountered or end-of-file.

\begin{verbatim}
@multiply (i1 i2):
    MUL i1 i2 -> i1
    RETURN i1

@getName ():
    LOAD @name -> w1
    RETURN w1
\end{verbatim}

Custom calling conventions and stack management may
be possible with future additions to the language.
